\section{Approximate Analytical Solution}

Structurally, the model described in Section \ref{sec:Model_Description} consists of a set of coupled ordinary differential equations with algebraic constraints for the economic production process and a stochastic adaptive network process for the social learning component.
We aim to find an analytic description of the dynamics of the model in terms ordinary differential equations of aggregated variables. This can be done in three steps. First, solve the algebraic constraints to the economic production process given by market clearing in the labor market and efficient production in the dirty sector, second use a moment closure to approximate the capital holdings of the heterogeneous households by the moments of their distribution and third use a pair approximation to describe the social learning process in terms of aggregated variables. Finally take the limit of infinitely many households (large system- or thermodynamic limit) to obtain a deterministic description of the system.

\subsection{Algebraic Constraints}
\JJK{Most of this subsection should probably go to sublementary material, depending on the journal requirements}

To calculate labor shares $L_c$ and $L_d$ as well as wages in the two sectors, we use equations \eqref{eq:resource_cost} and \eqref{eq:equilibrium_wage} and assume equal labor elasticities in both sectors $\alpha_d = \alpha_c = \alpha$ resulting in
\begin{align}
	w &= \frac{\partial Y_d}{\partial L_d} - \frac{\partial c_R}{\partial L_d} \nonumber \\
	&= \frac{\partial Y_d}{\partial L_d} - \frac{\partial c_R}{\partial R} \frac{\partial R}{\partial L_d} \nonumber = \frac{\partial Y_d}{\partial L_d} - \frac{\partial c_R}{\partial R} \frac{\partial}{\partial L_d} \frac{Y_d}{e} \nonumber \\
	&= \frac{\partial Y_d}{\partial L_d} - b_R \frac{\partial}{\partial L_d} \frac{Y_d}{e} = b_d K_d^{\beta_d} \alpha L_d^{\alpha-1}\left( 1-\frac{b_R}{e} \right)
	\label{eq:dirty_wages}
\end{align}
for the dirty sector and
\begin{equation}
	w = b_c K_c^{\beta_c} \alpha L_c^{\alpha-1}
	\label{eq:clean_wages}
\end{equation}
for the clean sector. Combining these results via equation \eqref{eq:population} results in
\begin{equation}
	L = \left( \frac{w}{\alpha} \right)^{\frac{1}{\alpha-1}}\left( \left( b_c K_c^{\beta_c} \right)^{\frac{1}{1-\alpha}} + \left( b_d K_d^{\beta_d} \left( 1 - \frac{b_R}{e} \right)^{\frac{1}{1-\alpha}} \right) \right).
\end{equation}
Substituting 
\begin{equation}
	X_c = (b_c K_c^{\beta_c}C^{\gamma})^{\frac{1}{1-\alpha}}, \qquad X_d = (b_d K_d^{\beta_d})^{\frac{1}{1-\alpha}}, \qquad X_R = \left( 1 - \frac{b_R}{e} \right)^{\frac{1}{1-\alpha}}
	\label{eq:substitutions}
\end{equation}
and solving for $w$ holds:
\begin{equation}
	w = \alpha L^{\alpha-1}\left( X_c + X_d X_R \right)^{1-\alpha}.
	\label{eq:wage_result}
\end{equation}
Plugging into equations \eqref{eq:dirty_wages} and \eqref{eq:clean_wages} results in 
\begin{align}
	L_c &= L \frac{X_c}{X_c + X_d X_R} \label{eq:clean_labor} \\
	L_d &= L \frac{X_d X_R}{X_c + X_d X_R} \label{eq:dirty_labor}
\end{align}
for the labor shares and plugging this into \eqref{eq:resources} results in
\begin{equation}
	R = \frac{b_d}{e}K_d^{\beta_d}L^{\alpha}\left( \frac{X_d X_R}{X_c + X_d X_R} \right)^{\alpha}
	\label{eq:R_result}
\end{equation}
for the use of the fossil resource. Using the results for $L_c$ and $L_d$ together with equations \eqref{eq:clean_capital_rent} and \eqref{eq:dirty_capital_rent}, the capital rental rates result in
\begin{align}
	r_c &= \frac{\beta_c}{K_c}X_c L^{\alpha}\left( X_c + X_d X_R \right)^{-\alpha}, \label{eq:r_c_result}\\
	r_d &= \frac{\beta_d}{K_d}X_d X_R L^{\alpha}\left( X_c + X_d X_R \right)^{-\alpha}. \label{eq:r_d_result}
\end{align}
It is also worth noting, that the assumption of zero profits, e.g.
\begin{align}
	Y_c &= w L_c + r_c K_c \nonumber \\
	Y_d &= w L_d + r_d K_d + c_R \nonumber
\end{align}
results in the following restraints for the capital and labor elasticities $\alpha$, $\beta_c$ and $\beta_d$:
\begin{equation}
	\beta_c = \beta_d = 1-\alpha.
	\label{eq:elasticities_restriction}
\end{equation}
To sum up, we solved the algebraic constraints to the ordinary differential equations describing the economic production process resulting in the following equations:
\begin{subequations}
\begin{empheq}[box=\widefbox]{gather}
	X_c = (b_c K_c^{\beta_c}C^{\gamma})^{\frac{1}{1-\alpha}}, \quad X_d = (b_d K_d^{\beta_d})^{\frac{1}{1-\alpha}}, \quad X_R = \left( 1 - \frac{b_R}{e}\frac{G_0^2}{G^2} \right)^{\frac{1}{1-\alpha}}, \\
	w = \alpha L^{\alpha-1}\left( X_c + X_d X_R \right)^{1-\alpha}, \label{eq:equilibrium_wage_solution}\\
	r_c = \frac{\beta_c}{K_c}X_c L^{\alpha}\left( X_c + X_d X_R \right)^{-\alpha}, \\
	r_d = \frac{\beta_d}{K_d}X_d X_R L^{\alpha}\left( X_c + X_d X_R \right)^{-\alpha}, \\
	R = \frac{b_d}{e}K_d^{\beta_d}L^{\alpha}\left( \frac{X_d X_R}{X_c + X_d X_R} \right)^{\alpha}, \\
	\dot{G} = - R \label{eq:sum_up_resource_depletion}, \\ 
	\dot{K}_c^{(i)} = s \delta(o_i - c) (r_c K_c^{(i)} + r_d K_d^{(i)} + w L^{(i)}) - \delta K_c^{(i)}, \label{eq:clean_capital_accumulation} \\
	\dot{K}_d^{(i)} = s \delta(o_i - d) (r_c K_c^{(i)} + r_d K_d^{(i)} + w L^{(i)}) - \delta K_d^{(i)}, \label{eq:dirty_capital_accumulation} \\
	\dot{G} = - R, 
	\label{eq:resource_depreciation}
\end{empheq}
\end{subequations}

\subsection{Moment Closure}

To describe the capital structure in the model, we use the numbers of households investing in clean and dirty capital $N^{(c)}$ and $N^{(d)}$ and look at the averages of their respective capital holdings that in the limit of $N \rightarrow \infty$ converge to the first moments of their respective distribution:
\begin{align}
	\bar{K}_l^{(k)} = \frac{1}{N^{(k)}}\sum_{o_i=k}^{N}K_l^{(i)}, \qquad \lim_{N \rightarrow \infty} \bar{K}_{l}^{(k)} = \braket{K_l^{(i)}}{o_i = k} = \mu_l^{(k)}
	\label{eq:moments_definition}
\end{align}
The time derivative of the means defined in \eqref{eq:moments_definition} is given by capital accumulation \eqref{eq:clean_capital_accumulation} and \eqref{eq:dirty_capital_accumulation} as well as terms resulting from agents switching their savings decisions. 
\begin{equation}
\left.  \begin{aligned}
		\dot{\bar{K}}_c^{(c)} &= (sr_c - \alpha)\bar{K}_c^{(c)} + s r_d \bar{K}_d^{(c)} + s w \bar{L} \nonumber \\
		\dot{\bar{K}}_d^{(c)} &= - \alpha\bar{K}_d^{(c)} \nonumber \\
		\dot{\bar{K}}_c^{(d)} &= - \alpha\bar{K}_c^{(d)} \nonumber \\
		\dot{\bar{K}}_d^{(d)} &= sr_c \bar{K}_c^{(d)} + (s r_d - \alpha)\bar{K}_d^{(d)} + s w \bar{L} \nonumber
	\end{aligned} \right\} + \rm{switching\ terms}
\end{equation}
The form of the switching terms can be calculated by taking the derivative of the moments as defined in \eqref{eq:moments_definition}:
\begin{align}
	\frac{d}{dt}\bar{K}_j^{(i)} =& \frac{d}{dt}\frac{1}{N^{(i)}}\sum_{\sigma_i = k}^{N} \bar{K}_j^{(i)} \\
	=& - \sum_{\sigma_i = k}^{N} \bar{K}_j^{(i)} \left(\frac{1}{N^{(i)}}\right)^{2}\frac{d}{dt} N^{(i)} \nonumber \\
	&+ \frac{1}{N^{(i)}} \left( \sum_{\sigma_i = k}^{N} \left. \frac{d}{dt} \right|_{N^{(i)}}\bar{K}_j^{(i)} + \frac{d N_{l \rightarrow i}}{dt} \bar{K}_{j}^{l} - \frac{d N_{i \rightarrow l}}{dt} \bar{K}_{j}^{(i)} \right) \\
	=& - \bar{K}_{j}^{(i)}\frac{1}{N^{(i)}}\left( \frac{d N_{l \rightarrow i}}{dt} - \frac{d N_{i \rightarrow l}}{dt} \right) \nonumber \\
	&+ \left. \frac{d}{dt} \right|_{N^{(i)}}\bar{K}_j^{(i)} +  \frac{1}{N^{(i)}} \left(\frac{d N_{l \rightarrow i}}{dt} \bar{K}_{j}^{l} - \frac{d N_{i \rightarrow l}}{dt} \bar{K}_{j}^{(i)} \right) \\
	=& \left. \frac{d}{dt} \right|_{N^{(i)}}\bar{K}_{j}^{(i)} + \frac{1}{N^{(i)}} \frac{dN_{l \rightarrow i}}{dt} \left(\bar{K}^{(l)}_{j} - \bar{K}^{(i)}_{j}\right) 
	\label{eq:mean_time_derivative}
\end{align}
which intuitively interprets as change in the mean capitatal through the influx of households from a different investment decision holding a different mean capital. They only depend on the different mean capital endowments held by households with different investment decision and the switching rates between investment decisions.

We derive expressions for these switching rates below.
\subsection{Pair Approximation}

To derive a macroscopic approximation of the opinion formation process, we make use of a Pair Based Proxy (PBP) that is derived via pair approximation from the adaptive network process. This is equivalent to describing the microscopic process in terms of aggregated quantities by making certain assumptions about the properties of their microscopic structure. The aggregated quantities of interest are the number of households investing in clean capital $N^{(c)}$, the number of households investing in dirty capital $N^{(d)}$ as well as the number of links between agents of the same group $[cc]$ and $[dd]$ as well as between the two groups $[cd]$. Since the total number of households $N$ and links $M$ is fixed, these five variables reduce to three degrees of freedom:

\begin{equation}
	X = N^{(c)} - N^{(d)}, \quad Y = [cc] - [dd], \quad Z = [cd]
	\label{eq:opinion_formation_macro_variables}
\end{equation}

These three degrees of freedom span the reduced state space of the social process $\mathbf{S} = (X, Y, Z)^T$. The investment decision making process can be described in terms of jump lengths $\Delta \mathbf{S}_j$ and jump rates $W(\mathbf{S},\mathbf{S} + \Delta \mathbf{S}_j)$ in this state space for the set $\Omega$ of all possible events.
The derivation of these is illustrate by the example of a clean household imitating a dirty household: The rate of this event is given by
\begin{equation}
	W_{c \rightarrow d} = \frac{N}{\tau} (1-\varepsilon) (1 - \varphi) \frac{N^{(c)}}{N}\frac{[cd]}{[cd] + 2 [cc]}\frac{1}{2}\left( \tanh(\bar{F}^{(d)} - \bar{F}^{(d)}) + 1 \right).
	\label{cdswitchingprob}
\end{equation}
In some more detail this results from
\begin{itemize}
	\item $N/\tau$ the rate of social update events i.e. the rate of events per household times the number of household,
	\item $(1-\varepsilon)$ the probability of the event not being a noise event,
	\item $(1-\varphi)$ the probability of imitation events (versus network adaptation events),
	\item $N^{(c)}/N$ the probability of the active household to invest in clean capital,
	\item $[cd]/(2[cc] + [cd])$ the probability of interaction with a household investing in dirty capital,
	\item $\frac{1}{2}\left( \tanh(\bar{F}^{(d)} - \bar{F}^{(c)}) + 1 \right)$ the probability of the active neighbor imitating its neighbor depending on the average difference in consumption between households investing in clean and dirty capital.
\end{itemize}
The change in state space variables is a little more tricky. Since the event is a clean household imitating a dirty household, we already know about one of the neighbors of the household. Then the remaining neighbors are determined by drawing $k^{c} - 1$ times from the distribution of neighbors that is assumed to be given by the probability for a neighbor to be dirty $p^{(d)}$ or clean $p^{(c)}$ where $k^{(c)}$ is the mean degree, e.g. the mean number of neighbors of a household in the network:
\begin{equation}
	p^{(c)} = \frac{2 [cc]}{2[cc] + [cd]}; \qquad p^{(d)} = \frac{[cd]}{2[cc] + [cd]}
	\label{neighbordist}
\end{equation}
Which results in $n^{(c)}$ additional clean neighbors and $n^{(d)}$ additional dirty neighbors.
\begin{equation}
	n^{(c)} = (1-1/k^{(c)})\frac{2[cc]}{N^{(c)}}; \quad n^{(d)} = (1-1/k^{(c)})\frac{[cd]}{N^{(c)}}
	\label{additional_neighbors}
\end{equation}
This results in the following changes in the state space variables:
\begin{align}
	\Delta N^{(c)} &= -1 \nonumber \\
	\Delta N^{(d)} &= 1 \nonumber \\
	\Delta [cc] &= \left( 1 - \frac{1}{k^{(c)}} \right)\frac{2[cc]}{N^{(c)}} \nonumber \\
	\Delta [dd] &= \left( 1 - \frac{1}{k^{(c)}} \right)\frac{[cd]}{N^{(c)}} \nonumber \\
	\Delta [cd] &= -1 + \left( 1 - \frac{1}{k^{(c)}} \right)\frac{2[cc] - [cd]}{N^{(c)}} \nonumber
\end{align}
So, summing up, the change in the state vector is given by:
\begin{equation}
	\Delta S_{c \rightarrow d} = \colvec{3}{-2}{-k^{(c)}}{-1 +  \left( 1 - \frac{1}{k^{(c)}} \right)\frac{2[cc] - [cd]}{N^{(c)}} }
	\label{cdstatespacechange}
\end{equation}

In terms of these, the dynamics of the PBP can be written as a master equation for the probability distribution $P$ over the state space $\mathbf{S}$:

\begin{align}
	\frac{{\partial} P(\mathbf{S}, t)}{\partial t} = \sum_{j \in \Omega} &P(\mathbf{S} - \Delta \mathbf{S}_j, t) W(\mathbf{S} - \Delta \mathbf{S}_j,\mathbf{S}) \nonumber \\
	&- P(\mathbf{S}, t) W(\mathbf{S},\mathbf{S} + \Delta \mathbf{S}_j) \label{eq:PBP}
\end{align}
\JJK{This illustrative derivation may as well go into the SI to be replaced by a short verbal explanation of the process in the paper body}

\subsection{Large System Limit}
\label{large_system_limit}
The description of the model in terms of equations \eqref{eq:equilibrium_wage_solution} to \eqref{eq:resource_depreciation} and \eqref{eq:PBP} pose a significant reduction of complexity, yet they are still not a description in therms of ordinary differential equations, as typically used in macroeconomic models. To derive ordinary differential equations, we do an expansion in terms of system size, which in our case is given by the number of households $N$.
Doing this, one has to make a choice: either A) keep the aggregate quantitites for labor, capital and resource stock fixed and consequently take the limit of infinitely many, infinitely small households, or B) keep the per household quantities of labor, capital and resource stock fixed and thereby go to an infinitely large economy in terms of aggregated quantitites. \\
\subsubsection*{Fixed per Household quantities}
First, we derive ordinary differential equations using option B). Therefore, following van Kampen, we Introduce rescaled variables
\begin{equation}
	x = \frac{X}{N}, \quad y = \frac{Y}{M}, \quad z = \frac{Z}{M}, \quad c = \frac{C}{N}, \quad g = \frac{G}{N}, \quad k = \frac{2M}{N}.
	\label{eq:rescalled_pbp_variables}
\end{equation}
and expand the master equation in terms of a small parameter $N^{-1}$. In leading order, the time development of the rescaled state vector $\mathbf{s} = (x, y, z)$ is given by 
\begin{equation}
	\frac{\rm{d}}{\rm{d}t}\mathbf{s} = \alpha_{1,0}(\mathbf{s})
	\label{macroscopic_equation}
\end{equation}
where $\alpha_{1,0}$ is the first jump moment of $W$ in terms of rescalled variables $\mathbf{s}$ given by
\begin{equation}
	\alpha_{1,0}(\mathbf{s}) = \int \Delta \mathbf{s} W(s, \Delta \mathbf{s}) {\rm d} \Delta \mathbf{s}
	\label{eq:jump_moment}
\end{equation}
which in the case of discrete jumps in phase space simplifies to:
\begin{equation}
	\frac{\rm{d}}{\rm{d}t}\mathbf{s} = \sum_{j \in \Omega}  \Delta \mathbf{s_j} W_j 
	\label{eq:transitions}
\end{equation}
Where $\Omega$ is the set of all possible (discrete) events in the opinion formation process.
For the switching terms in \eqref{eq:mean_time_derivative} simply collect all terms describing immitation processes in the pair based proxy and sum up.
Rescalled variables in the economic system are:
\begin{equation}
	l = \frac{L}{N}, \quad g = \frac{G}{N}, \quad c = \frac{C}{N}.
	\label{eq:rescalled_economic_variables}
\end{equation}
Note that the economic equations \eqref{eq:equilibrium_wage_solution} to \eqref{eq:resource_depreciation} in terms of rescalled variables \eqref{eq:rescalled_pbp_variables} and \eqref{eq:rescalled_economic_variables} are only independent of $N$, if we assume constant returns to scale. This is therefore indispensable.

\subsubsection*{Fixed Aggregated Quantitites}
If we want to abstract from the restriction of constant returns to scale, we have to keep the aggregated economic quantities fixed in the large system limit. This means that instead of the means of capital endowmends of clean and dirty households, we have to use the sum over their endowmends as variables:
\begin{equation}
	\tilde{K_l^{(k)}} = \sum_{o_i=k}^{N}K_l^{(i)}
	\label{eq:not_normalized_moments_definition}
\end{equation}
and look at their time development:
\begin{equation}
	\dot{\tilde{K}}_{l}^{(k)} = \left. \frac{d}{dt} \right|_{N^{(k)}}\tilde{K}^{(i)}_{l} + \frac{d N_{j \rightarrow k}}{dt}\frac{\tilde{K}_l^{(j)}}{N^{(j)}} - \frac{d N_{k \rightarrow j}}{dt}\frac{\tilde{K}_l^{(k)}}{N^{(k)}}.
	\label{eq:aggregated_capital_time_derivative}
\end{equation}
In the limit of $N \rightarrow \infty$ the per household quantities vanish. Therefore, the interaction defined in eq. \eqref{eq:imitation_probability} that depends on absolute differences in household properties would vanish too. Consequently, we have to redefine the interaction to only depend on relative differences in household quantities instead.
\par
\JJK{Note: Agent interaction depending on absolute differences in per household quantities and non-constant returns to scale are mutually exclusive in the large system limit.}
