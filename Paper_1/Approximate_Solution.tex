\section{Approximate Analytical Solution}

Structurally, the model described in \ref{Model_Description} consists of a set of coupled ordinary differential equations with algebraic constraints for the economic production process and a stochastic adaptive network process for the social learning component.
We aim to find a description of the dynamics of the model in terms ordinary differential equations of aggregated variables. This can be done in three steps. First, solve the algebraic constraints to the economic production process given by market clearing in the labor market and efficient production in the dirty sector, second use a moment closure to approximate the capital holdings of the heterogeneous households by the moments of their distribution and third use a pair approximation to describe the social learning process in terms of aggregated variables.

\subsection{Algebraic Constraints}

To calculate labor shares $L_c$ and $L_d$ as well as wages in the two sectors, we use equations \eqref{equilibrium_wage} resulting in
\begin{align}
	w &= \frac{\partial Y_d}{\partial L_d} - \frac{\partial c_R}{\partial L_c} \nonumber \\
	&= \frac{\partial Y_d}{\partial L_d} - \frac{\partial c_R}{\partial R} \frac{\partial R}{\partial L_d} \nonumber = \frac{\partial Y_d}{\partial L_d} - \frac{\partial c_R}{\partial R} \frac{\partial}{\partial L_d} \frac{Y_d}{e} \nonumber \\
	&= \frac{\partial Y_d}{\partial L_d} - b_R \frac{\partial}{\partial L_d} \frac{Y_d}{e} = b_d K_d^{\beta_d} \alpha L_d^{\alpha-1}\left( 1-\frac{b_R}{e} \right)
	\label{dirty_wages}
\end{align}
for the dirty sector and
\begin{equation}
	w = b_c K_c^{\kappa_c} \pi P_c^{\pi-1}
	\label{clean_wages}
\end{equation}
for the clean sector. Combining these results via equation \eqref{population} results in
\begin{equation}
	P = \left( \frac{w}{\pi} \right)^{\frac{1}{\pi-1}}\left( \left( b_c K_c^{\kappa_c} \right)^{\frac{1}{1-\pi}} + \left( b_d K_d^{\kappa_d} \left( 1 - \frac{b_R}{e} \right)^{\frac{1}{1-\pi}} \right) \right)
\end{equation}
substituting 
\begin{equation}
	X_c = (b_c K_c^{\kappa_c})^{\frac{1}{1-\pi}}, \qquad X_d = (b_d K_d^{\kappa_d})^{\frac{1}{1-\pi}}, \qquad X_R = \left( 1 - \frac{b_R}{e} \right)^{\frac{1}{1-\pi}}
	\label{substitutions}
\end{equation}
holds the following result for $w$:
\begin{equation}
	w = \pi P^{\pi-1}\left( X_c + X_d X_R \right)^{1-\pi}.
	\label{wage_result}
\end{equation}
Plugging this into equations \eqref{dirty_wages} and \eqref{clean_wages} results in 
\begin{align}
	P_c &= P \frac{X_c}{X_c + X_d X_R} \label{clean_labor} \\
	P_d &= P \frac{X_d X_R}{X_c + X_d X_R} \label{dirty_labor}
\end{align}
and plugging this into \eqref{resources} results in
\begin{equation}
	R = \frac{b_d}{e}K_d^{\kappa_d}P^{\pi}\left( \frac{X_d X_R}{X_c + X_d X_R} \right)^{\pi}.
	\label{R_result}
\end{equation}
Using the results for $P_c$ and $P_d$ together with equations \eqref{wages_and_rents}, the capital rental rates result in
\begin{align}
	r_c &= \frac{\kappa_c}{K_c}X_c P^{\pi}\left( X_c + X_d X_R \right)^{-\pi}, \label{r_c_result}\\
	r_d &= \frac{\kappa_d}{K_d}X_d X_R P^{\pi}\left( X_c + X_d X_R \right)^{-\pi}. \label{r_d_result}
\end{align}
It is also worth noting, that the assumption of zero profits, e.g.
\begin{align}
	Y_c &= w P_c + r_c K_c \nonumber \\
	Y_d &= w P_d + r_d K_d + c_R \nonumber
\end{align}
results in the following restraints for the capital and labor elasticities $\pi$, $\kappa_c$ and $\kappa_d$:
\begin{equation}
	\kappa_c = \kappa_d = 1-\pi.
	\label{elasticities_restriction}
\end{equation}
\textbf{Equations that are actually implemented are:}
\begin{subequations}
\begin{empheq}[box=\widefbox]{gather}
	X_c = (b_c K_c^{\kappa_c})^{\frac{1}{1-\pi}}, \qquad X_d = (b_d K_d^{\kappa_d})^{\frac{1}{1-\pi}}, \qquad X_R = \left( 1 - \frac{b_R}{e}\frac{G_0^2}{G^2} \right)^{\frac{1}{1-\pi}}, \\
	w = \pi P^{\pi-1}\left( X_c + X_d X_R \right)^{1-\pi}, \\
	r_c = \frac{\kappa_c}{K_c}X_c P^{\pi}\left( X_c + X_d X_R \right)^{-\pi}, \\
	r_d = \frac{\kappa_d}{K_d}X_d X_R P^{\pi}\left( X_c + X_d X_R \right)^{-\pi}, \\
	R = \frac{b_d}{e}K_d^{\kappa_d}P^{\pi}\left( \frac{X_d X_R}{X_c + X_d X_R} \right)^{\pi}, \\
	\dot{K}_i^{(c)} = s \delta(x_i - c) (r_c K_i^{(c)} + r_d K_i^{(d)} + w P_i) - \delta K_i^{(c)}, \\
	\dot{K}_i^{(d)} = s \delta(x_i - d) (r_c K_i^{(c)} + r_d K_i^{(d)} + w P_i) - \delta K_i^{(d)}, \\
	\dot{G} = - R, 
\end{empheq}
\end{subequations}


\subsection{Moment Closure}

\subsection{Pair Approximation}


