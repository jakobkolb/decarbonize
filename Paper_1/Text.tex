

\section{Abstract}

Macroeconomic models formulate dependencies between aggregate economic variables. Since Lucas' critique in the 70s, is has become the dominating paradigm that to be credible, such models have to be built upon microfoundations. However, for the sake of mathematical traceability, even contemporary DSGE models commonly formulate their microfoundation assuming representative agents or, if they model heterogeneous agents, assuming mutually independent rational decision makers. \\
Yet, besides the fact that representative agent modeling is prone to fallacy of composition, assuming independent individuals is generally far off realities in which individual behavior is strongly influenced by social processes. \\
We show, that methods from statistical physics allow for the formulation of dependencies between aggregate economic variables, that are literally derived from an agent based model with interdependent social learners. 
We analyze effects of the assumptions underlying the aggregation process and evaluate them by comparison to results from numerical simulations as well as the trivial assumption of one representative household. \\
This shall illustrate that agent based modeling and analytical work in macroeconomics are by no means antithetical, but that there are methods with the potential to connect the two branches of theory. \\

\section{Contents}
\begin{itemize}
	\item Recapitulate Lucas' critique on macroeconomics without microfoundations as well as Kirmans critique of the representative individual,
	\item Elaborate on the strengths and weaknesses of agent based models,
	\item Explain/Link to particular statistical physics methods (moment closure, pair based approximation) useful to find aggregate solutions.
	\item Show prove of concept (explain the model and derive approximate solution)
	\item Discuss results: validity of solution and its dependence on the assumptions made during aggregation.
\end{itemize}