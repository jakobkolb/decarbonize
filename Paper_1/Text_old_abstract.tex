

\section{Abstract}

Macroeconomic models formulate dependencies between aggregate economic variables. Since Lucas' critique in the 70s, is has become the dominating paradigm that to be credible, such models have to be built upon microfoundations. However, for the sake of mathematical traceability, even contemporary DSGE models commonly formulate their microfoundation assuming representative agents or, if they model heterogeneous agents, assuming mutually independent rational decision makers. \\
Yet, besides the fact that representative agent modeling is prone to fallacy of composition, assuming independent individuals is generally far off realities in which individual behavior is strongly influenced by social processes. \\
We show, that methods from statistical physics allow for the formulation of dependencies between aggregate economic variables, that are literally derived from an agent based model with interdependent social learners. 
We analyze effects of the assumptions underlying the aggregation process and evaluate them by comparison to results from numerical simulations as well as the trivial assumption of one representative household. \\
This shall illustrate that agent based modeling and analytical work in macroeconomics are by no means antithetical, but that there are methods with the potential to connect the two branches of theory. \\

\newpage

\textbf{Guidelines on how to write an abstract from Nature:}\\

\emph{One sentence giving a very general introduction into the field:} \\

\emph{One to two more sentences giving some more detail, comprehensible to scientists in the field.} \\

\emph{One sentence clearly stating the general problem being addressed by this particular study.} \\

\emph{One sentence summarizing the main result (with the words ``here we show'' or their equivalent.)} \\

\emph{Two or three sentences explaining what the main result reveals in direct comparison to what was thought to be the case previously, or how the main results adds to previous knowledge.} \\

\emph{One or two sentences to put the result into a more general context.} \\

\emph{Two to three sentences to provide a broader perspective and/or outlook, readily comprehensible to a scientist in any discipline.} \\


\section{Abstract Old}

In recent years, the assumptions of independence and utility maximization underlying many classical macroeconomic models have been severely challenged.  \\
It is widely acknowledged that individuals economic decisions are only boundedly rational decision makers and that social learning and opinion formation processes may play an important role in the process. Nevertheless, reasoning in line with neoclassical economic theories is still widely used in the motivation of political decision making regarding economic matters. Therefore, we extend a neoclassical approach to include heterogeneous households whose individual behavior is grounded in imitation of successful behavior and homophily. \\
Although an algorithmic representation of rules for the behavior of microscopic agents can be formulated with ease, the transition from a microscopic to a macroscopic description remains a challenge. \\
Here we show that methods from statistical physics and adaptive network theory can be applied to derive a macroscopic approximation in therms of ordinary differential equations. \\
Previously, effects of heterogeneous agents were studied by introducing classes of agents. Here we show, that truly granular agents can be treated as well.\\
Considering higher moments of the capital distribution, this framework enables the study of income inequality in the context of a classical savings framework, providing a tool that might helpfully contribute to the ongoing inequality debate. \\
