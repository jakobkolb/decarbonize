\section{Model Description}
\label{sec:Model_Description}
\JJK{Maybe add a disclaimer about the model not being meant too serious but rather as an illustration of how social dynamics can be integrated yet with a hint to a contemporary problem/question}\par

The model consists of two processes. One for the economic production process and one for social learning about investment decisions amongst heterogeneous households.

\subsection{Economic production}
\label{sec:model_description}

Economic production takes place in two sectors. For illustrational clarity we call them the \textit{clean} and the \textit{dirty} sector. The dirty sector uses labor $L$, dirty capital $K_d$ and a fossil resource $R$ as input factors, the clean sector uses labor $L$, clean capital $K_c$ and a renewable technology stock $C$ as input factors. We assume that the physical capital of one sector cannot be used in the other sector due to different technologies. We also assume that capital labor and knowledge are mutual substitutes but that the additional resource cannot be substituted. To satisfy these requirements, we use the following production functions:
\begin{equation}
	Y_c = b_c L_c^{\alpha_c}K_c^{\beta_c}C^{\gamma}, 
	\label{eq:clean_production}
\end{equation}
in the clean sector, where $b_c$ is the Solow residual in the clean sector and $\alpha_c$, $\beta_c$ and $\gamma_c$ are the elasticities of the respective input factors and
\begin{equation}
	Y_d = {\rm min}\left( b_d L_d^{\alpha_d}K_d^{\beta_d}, e R \right),
	\label{eq:dirty_production}
\end{equation}
in the dirty sector, where $b_d$ is the Solow residual in the dirty sector, $\alpha_d$ and $\beta_d$ are the input factors of the respective input factors and $e$ is the conversion efficiency of the fossil resource.
Additionally, it is assumed that
\begin{itemize}
	\item the usage of the fossil resource $R$ depletes a geological resource stock $G$:
		\begin{equation}
			\dot{G} = -R, \quad G(t=0) = G_0
			\label{eq:resource_depletion}
		\end{equation}
	\item the cost for the usage of the fossil resource depends on the resource use $R$ and the fraction of the remaining fossil resource stock $G/G_0$:
		\begin{equation}
			c_R = b_R R^{\rho} = \tilde{b}_R R^{\rho}\left( \frac{G}{G_0} \right)^{\mu}
			\label{eq:resource_cost}
		\end{equation}
	\item perfect labor mobility and competition for labor between the two sectors, leading to an equilibrium wage that equals the marginal return for labor:
		\begin{equation}
			w = \frac{\partial Y_c}{\partial L_c} = \frac{\partial Y_d}{\partial L_d} - \frac{\partial c_R}{\partial L_d}
			\label{eq:equilibrium_wage}
		\end{equation}
		where the sum of the labor shares equals the total amount of labor available:
		\begin{equation}
			L_c + L_d = L.
			\label{eq:population}
		\end{equation}

	\item technology specific capital, that can only be used in the sector that it has initially been invested in, leading to independent capital return rates for the two kinds of capital:
		\begin{align}
			r_c &= \frac{\partial Y_c}{\partial K_c} \label{eq:clean_capital_rent}\\
			r_d &= \frac{\partial Y_d}{\partial K_d} - \frac{\partial c_R}{\partial K_d} \label{eq:dirty_capital_rent}
		\end{align}
	\item learning by doing with a chance of forgetting in the clean sector, leading to change in the renewable technology stock $C$:
		\begin{equation}
			\dot{C} = Y_c - \chi C
			\label{learning_by_doing}
		\end{equation}
	\item Lastly, we assume efficient allocation of resources in the dirty sector which is equivalent to
		\begin{equation}
			e R = b_d L^{\alpha_d}K^{\beta_d}.
			\label{eq:resources}
		\end{equation}
\end{itemize}

\subsection{Investment Decision Making}

There are $N$ heterogeneous households denoted with the index $i$ as owners of labor $L^{(i)}$ and capital $K_c^{(i)}$ and $K_d^{(i)}$ in (potentially) both economic sectors.
Households generate income from labor and capital income:
\begin{equation}
	I^{(i)} = w L^{(i)} + r_c K_c^{(i)} + r_d K_d^{(i)}
	\label{household_income}
\end{equation}
which they use for consumption $F^{(i)} = (1-s) I^{(i)}$ and savings $S^{(i)} = s I^{(i)}$. Households decision about where to invest their savings $o_i \in [c,d]$ is the result of a social learning process closely adapted from the adaptive voter model. This process describes the learning of successful strategies together with preferential attachment driven by homophily towards individuals exhibiting the same behavior.

Households form the nodes of a graph of acquaintance relations. Households get active at a constant rate. When a household $i$ becomes active, it interacts with one of its acquaintances $j$ chosen at random. If they hold the same opinion, nothing happens. If they hold different opinions, one of two things can happen:
\begin{itemize}
	\item Adaptation: with probability $\varphi$, the households end their relation and household $i$ connects to another household $k$, that holds the same opinion. 
	\item Imitation: with probability $1-\varphi$, an imitation event takes place and household $i$ assumes the opinion of household $j$ with probability $p_{ji}$ increasing with their difference in income.
\end{itemize}
The imitation probability is given as a monotonously increasing function of the difference in consumption between the two households:
\begin{equation}
	p_{ji} = \frac{1}{2}(1 + \tanh \left( F^{(j)} - F^{(i)}  \right)
	\label{eq:imitation_probability}
\end{equation}
A fraction $\varepsilon$ of events are random, e.g. rewiring to a random other household or randomly investing in one of the two sectors.
\JJK{Elaborate on why one should implement random events here?}

Given the savings decisions of the individual households, the time development of their capital holdings is given by

\begin{align}
	\dot{K}_c^{(i)} =& \delta_{o_i, c} \left( r_c K_c^{(i)} + r_d K_d^{(i)} + w L_i \right) - \kappa K_c \\
	\dot{K}_d^{(i)} =& \delta_{o_i, d} \left( r_c K_c^{(i)} + r_d K_d^{(i)} + w L_i \right) - \kappa K_d 
\end{align}

Assuming equal capital depreciation rates $\kappa$ in both sectors.
